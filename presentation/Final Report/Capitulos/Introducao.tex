
\chapter{Introdu\c{c}\~ao}

A emerg\^encia de ambientes de experimenta\c{c}\~ao, conhecidos como \textit{testbeds}, para o amparo do desenvolvimento de agentes conectatos em rede, cria a necessidade por protocolos e ferramentas que possibilitem a descri\c{c}\~ao dos recursos, da configura\c{c}\~ao e dos processos que definem completamente um experimento reproduz\'ivel. Neste contexto, desenvolveu-se o \textit{OMF} \cite{Rakotoarivelo2010:OCM:1713254.1713267}, um arcabou\c{c}o para ger\^encia de \textit{testbeds} e controle de experimentos.

O objetivo do OMF \'e assegurar o n\'ivel apropriado de abstra\c{c}\~ao, tanto do ponto de vista do operador de \textit{testbeds} quanto do pesquisador \cite{OMFExecutiveSummary}. O operador tem acesso a um conjunto de servi\c{c}os que facilitam a ger\^encia do \textit{testbed}. Por outro lado, o arcabou\c{c}o oferece suporte a \textit{scripts}, escritos pelo pesquisador em linguagem de alto n\'ivel, que descrevem o experimento a ser realizado, automatizando sua execu\c{c}\~ao.

O LabLua desenvolveu, nos \'ultimos anos, pesquisas sobre redes de sensores sem fio (RSSF) e participou do desenvolvimento do \textit{testbed} CeuNaTerra \cite{rossettoceunaterra}, implementado com o apoio da Rede Nacional de Ensino e Pesquisa. Também desenvolveu um conjunto de tecnologias que podem ser empregadas em aplica\c{c}\~oes destinadas a RSSF, como a linguagem de programa\c{c}\~ao estruturada s\'incrona e reativa C\'eu \cite{ceu.tr} e o sistema de programa\c{c}\~ao de RSSF Terra \cite{Branco:2015:TFS:2782756.2811267}. 

Neste contexto, surgiu a ideia de se expandir os casos de uso da linguagem C\'eu atrav\'es do desenvolvimento nesta de uma implementa\c{c}\~ao alternativa do arcabou\c{c}o OMF. Isto se deve a experimentos OMF serem descritos em torno de eventos externos e internos aos agentes sob teste e a exist\^encia de construtos nativos na linguagem para defini\c{c}\~ao e tratamento de eventos como entidades de primeira classe.

% \begin{figure}
%   \includegraphics*[width=\linewidth]{ctor4_none.eps}
%   \caption{Uma figura}
% \end{figure}