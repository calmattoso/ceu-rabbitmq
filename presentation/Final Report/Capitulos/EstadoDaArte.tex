
\chapter{Situa\c{c}\~ao Atual}

O arcabou\c{c}o OMF pode ser discretizado em tr\^es planos: controle, medi\c{c}\~ao e gerenciamento \cite{Rakotoarivelo2010:OCM:1713254.1713267}. O primeiro engloba as ferramentas e diretivas que possibilitam ao pesquisador definir e por em execu\c{c}\~ao um experimento; aqui destaca-se o \textit{controlador de experimentos (CE)}. O segundo \'e constitu\'ido das tecnologias que possibilitam ao pesquisador coletar m\'etricas dos recursos sob teste para avalia\c{c}\~ao posterior; a coleta de m\'etricas \'e definida atrav\'es da \textit{OMF Measurement Library (OML)}, n\~ao abordada neste trabalho. Finalmente, a camada de gerenciamento diz respeito a infraestrutura e \textit{testbed} nos quais o experimento \'e executado, destacando-se neste plano os recursos e seus controladores.

Os recursos na camada de gerenciamento s\~ao os alvos do experimento. Para possibilitar o desenvolvimento de tais recursos como m\'odulos coesos e independentes, o arcabou\c{c}o OMF define entidades denominadas \textit{controladores de recursos (CRs)}, que s\~ao respons\'aveis por intermediar a comunica\c{c}\~ao entre o CE e os recursos sob seu controle. A distribui\c{c}\~ao oficial do OMF disponibiliza uma biblioteca em \textit{Ruby}, chamada de \textit{omf\_rc} \cite{omf.rc}, que permite a descri\c{c}\~ao e uso de controladores de recursos. Contudo, controladores de recursos podem ser implementados sem depender desta biblioteca, desde que lidem corretamente com os protocolos de comunica\c{c}\~ao.

O CE na camada de controle \'e o aplicativo que executa o experimento, sendo sua implementa\c{c}\~ao oficial chamada de \textit{omf\_ec} \cite{omf.ec}. Para executar um experimento, o pesquisador deve fornecer como entrada para o CE um arquivo denominado \textit{descri\c{c}\~ao de experimento (DE)} que especifique completamente os recursos sob teste, sua configura\c{c}\~ao inicial e as a\c{c}\~oes a serem executadas sobre os mesmos durante o experimento. Com base neste arquivo, o CE envia pedidos aos CRs dos recursos sob teste e coleta as respostas enviadas por estes, exibindo-as ao usu\'ario caso este especifique tal comportamento. Na distribui\c{c}\~ao oficial do OMF, tais arquivos devem ser elaborados na linguagem de programa\c{c}\~ao \textit{OMF Experiment Description Language (OEDL)} \cite{OEDL}. Esta linguagem \'e, na verdade, uma extens\~ao de \textit{Ruby} que prov\^e uma s\'erie de comandos espec\'ificos ao dom\'inio de defini\c{c}\~ao e orquestra\c{c}\~ao de experimentos. Deste modo, o pesquisador pode facilmente descrever seus experimentos, mas tamb\'em tem acesso as funcionalidade nativas de \textit{Ruby}, sendo poss\'ivel realizar l\'ogicas mais complexa em seu DE.

A comunica\c{c}\~ao entre o CE e os CRs n\~ao se d\'a de forma direta. O arcabou\c{c}o OMF exige o emprego de um servidor que suporte o paradigma de troca de mensagens \textit{pub/sub}. Utilizando-se este servidor, ambas as partes trocam mensagens atrav\'es de t\'opicos definidos ao longo da execu\c{c}\~ao do experimento, segundo o protocolo \textit{Federated Resource Control Protocol (FRCP)} \cite{frcp}. Este protocolo especifica o formato do \textit{payload} das mensagens intercambiadas e uma s\'erie de regras sobre como o interc\^ambio destas deve ocorrer, segundo seu tipo e par\^ametros. A implementa\c{c}\~ao oficial de OMF suporta dois tipos de protocolos de troca de mensagens: \textit{Extensible Messaging and Presence Protocol (XMPP)} e \textit{Advanced Message Queuing Protocol (AMQP)} \cite{oasis2012advanced}, este mais recomendado.

AMQP especifica que a troca de mensagens entre aplica\c{c}\~oes deve ocorrer por interm\'edio de filas de mensagens atreladas a \textit{exchanges}, ambos definidos e gerenciados por um servidor central, chamado de \textit{broker}. Aplica\c{c}\~oes produtoras de mensagens as enviam para um \textit{exchange} que, com base em seu tipo e em uma chave de roteamento possivelmente definida no cabeçalho de cada mensagem, roteia cada uma para o subconjunto apropriado de filas ou para o descarte, caso nenhuma fila atrelada ao \textit{exchange} esperasse por mensagens com a chave especificada. Aplica\c{c}\~oes consumidoras, por sua vez, increvem-se em filas e consomem mensagens destas segundo um crit\'erio de \textit{round-robin}. No contexto de OMF deve-se utilizar um servidor que implemente o protocolo AMQP, através do qual o CE e CRs realizam sua troca de mensagens.

O ciclo de vida de um experimento inicia-se com a elabora\c{c}\~ao de uma descri\c{c}\~ao de experimento que \'e ent\~ao passada como entrada a um controlador de experimentos. Este, por sua vez, inicia uma sequ\^encia de troca de mensagens com os controladores de recursos especificados na DE. Esta troca de mensagens ocorre atrav\'es de um \textit{broker} AMQP, sendo a defini\c{c}\~ao dos t\'opicos, empacotamento das mensagens e roteamento destas realizados segundo as regras estipuladas pelo FRCP. Tais processos s\~ao executados por tr\'as dos panos pelo CE, sem a necessidade de conhecimento ou interven\c{c}\~ao do pesquisador sobre os mesmos. Ao fim do experimento, o estado no servidor de mensagens \'e limpo e poss\'iveis m\'etricas s\~ao armazenadas automaticamente.

Embora OMF seja um arcabou\c{c}o robusto, uma limita\c{c}\~ao inerente a \textit{Ruby} \'e a aus\^encia de eventos como entidades de primeira classe. Isto acarreta em defini\c{c}\~oes de eventos e diretivas para seu tratamento n\~ao t\~ao naturais, constituindo uma barreira para a introdu\c{c}\~ao de pesquisadores a este ambiente. Para facilitar o uso de \textit{testbeds} OMF, pretendemos oferecer a possibilidade do pesquisador escrever a descri\c{c}\~ao de seu experimento em C\'eu, uma linguagem desenvolvida para lidar com sistemas orientados a eventos.