
\chapter{Proposta e Objetivos do Trabalho}

O principal produto deste trabalho ser\'a a implementa\c{c}\~ao de um CE alternativo em C\'eu que apresente paridade de funcionalidades frente ao oficial e seja capaz de processar DEs escritas em C\'eu. A escrita de DEs em C\'eu tornar\'a poss\'ivel a defini\c{c}\~ao e tratamento de eventos como entidades de primeira classe, simplificando a estrutura desses arquivos. Al\'em disso,introduzir\'a um novo caso de uso de C\'eu sob o dom\'inio de defini\c{c}\~ao e orquestra\c{c}\~ao de experimentos em \textit{testbeds}. 

\'E necess\'aria a ado\c{c}\~ao de uma biblioteca de AMQP para que o CE C\'eu seja capaz de se comunicar com os CRs segundo o processo delineado no cap\'itulo anterior. Sua implementa\c{c}\~ao em C\'eu como um m\'odulo independente justifica-se por tornar mais conveniente seu emprego em futuras aplica\c{c}\~oes C\'eu. Esta biblioteca ser\'a implementada com base na biblioteca C \textit{rabbitmq-c} \cite{rabbitmq.c}, expondo as funcionalidades essenciais do AMQP por esta suportadas. Contudo, devido a natureza bloqueante da implementa\c{c}\~ao de certas funcionalidades na biblioteca C, propriedade esta contr\'aria ao paradigma s\'incrono reativo de C\'eu, ser\~ao impostas limita\c{c}\~oes ao uso de tais funcionalidades para que se ofere\c{c}a uma biblioteca de mais alto n\'ivel em C\'eu.

As bibliotecas desenvolvidas adotarão de forma o mais rigorosa quanto possível as imposições de Céu, em detrimento aos conceitos dos protocolos AMQP e OMF. Por exemplo, ao invés de se definir uma variável de estado em uma chamada AMQP que destrói uma entidade quando do término de uma conexão, será imposto ao programador que a destruição da entidade ocorra através do término do escopo no qual sua representação em Céu reside. Deste modo, impera o conceito de programação estruturada de Céu: o fluxo de execução do programa se dá segundo reações a eventos, as quais implicam a ativação ou terminação de escopos que, por sua vez, levam a criação e execução ou terminação de entidades (e.g. filas AMQP).

Para valida\c{c}\~ao dos produtos desenvolvidos, ser\'a disponibilizada uma s\'erie de testes que garanta a corretude das funcionalidades da biblioteca AMQP e do CE OMF C\'eu. Finalmente, se houver disponibilidade de tempo, tentar-se-á também neste trabalho apresentar uma demonstração de um experimento que execute sobre o \textit{testbed} CeuNaTerra atrav\'es de comandos publicados pelo CE desenvolvido em C\'eu. Ainda, para fins de comparaç\~ao, seria tamb\'em feita uma demonstra\c{c}\~ao equivalente utilizando o CE oficial.

Este trabalho n\~ao ir\'a oferecer uma integra\c{c}\~ao com OML, mas o c\'odigo final do CE dever\'a ser o mais claro e extens\'ivel poss\'ivel para que tal integra\c{c}\~ao possa vir a ser adicionada em trabalhos futuros. Al\'em disso, n\~ao ser\'a entregue uma biblioteca em C\'eu para implementa\c{c}\~ao de controladores de recursos.
