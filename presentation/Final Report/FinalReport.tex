\documentclass[graduacao,brazil]{ThesisPUC}


%%%%%%%%%%%%%%%%%%%%%%%%%%%%%%%%%%%%%%%%%%%%%%%%%%%%%%%%%%%%%%%%%%%%%%%%%%%%%%%%

\newcommand{\Rset}{\mathbb{R}}
\newcommand{\Zset}{\mathbb{Z}}

%%%%%%%%%%%%%%%%%%%%%%%%%%%%%%%%%%%%%%%%%%%%%%%%%%%%%%%%%%%%%%%%%%%%%%%%%%%%%%%%

\autor{Carlos Mattoso}
\autorR{Mattoso, Carlos}
\orientador{Noemi Rodriguez}
\orientadorR{Rodriguez, Noemi}

\titulo{Desenvolvimento de Controlador de Experimentos OMF em C\'{e}u}
\titulouk{Development of OMF Experiment Controller in C\'{e}u}

\subtitulo{}
\dia{21} \mes{Junho} \ano{2016}

\cidade{Rio de Janeiro}
\CDD{510}
\departamento{Inform\'atica}
\programa{Engenharia de Computa\c{c}\~{a}o}
\centro{Centro T\'{e}cnico Cient\'{i}fico}
\universidade{Pontif\'{i}cia Universidade Cat\'{o}lica do Rio de Janeiro}
\uni{PUC--Rio}
\course{Bachalerado em Engenharia de Computa\c{c}\~{a}o}
\diploma{Bacharel em Engenharia de Computa\c{c}\~{a}o}

%%%%%%%%%%%%%%%%%%%%%%%%%%%%%%%%%%%%%%%%%%%%%%%%%%%%%%%%%%%%%%%%%%%%%%%%%%%%%%%%

% Não precisa preencher se for um Projeto Final %
%
\banca{
  \membrodabanca{Luis Carlos Pacheco R. Velho}{IMPA}
  \membrodabanca{Jorge Stolfi}{UNICAMP}
  \coordenador{Ney Augusto Dumont}
}
%

%%%%%%%%%%%%%%%%%%%%%%%%%%%%%%%%%%%%%%%%%%%%%%%%%%%%%%%%%%%%%%%%%%%%%%%%%%%%%%%%

% Não precisa preencher se for um Projeto Final %
%
\curriculo{%
Graduou--se em Engenharia na Ecole Polytechnique (Paris, Fran\c{c}a), cursando \'{A}lgebra e Inform\'{a}tica, assim como F\'{i}sica Te\'{o}rica. Especializou--se na Ecole Sup\'{e}rieure des T\'{e}l\'{e}communications (Paris, Fran\c{c}a) em Processamento de Sinais de Voz e Imagens, assim como Organiza\c{c}\~{a}o e Planejamento. Trabalhou junto com a empresa Inventel em sistemas de telecomunica\c{c}\~{o}es sem fil baseados na tecnologia BlueTooth. Desenvolveu junto com os seus orientadores durante o Mestrado ferramentas de topologia computacional.}%
%


%%%%%%%%%%%%%%%%%%%%%%%%%%%%%%%%%%%%%%%%%%%%%%%%%%%%%%%%%%%%%%%%%%%%%%%%%%%%%%%%

% \epigrafe{%
% Lorem ipsum.
% }
% \epigrafeautor{Wassily Kandinsky}
% \epigrafelivro{Regarde}
\renewcommand{\epigrafe}{}

%%%%%%%%%%%%%%%%%%%%%%%%%%%%%%%%%%%%%%%%%%%%%%%%%%%%%%%%%%%%%%%%%%%%%%%%%%%%%%%%

% %
% \agradecimentos{%
% Aos meus orientadores Professores H\'{e}lio Lopes e Geovan Tavares pelo apoio, simpatia de sempre, e incentivo para a realiza\c{c}\~{a}o deste trabalho

% Ao CNPq e \`{a} PUC--Rio, pelos aux\'{i}lios concedidos, sem os quais este trabalho n\~{a}o poderia ter sido realizado.

% \`{A}s minhas av\'{o}s, que sofreram o mais pela saudade devida a minha expatria\c{c}\~{a}o. Aos meus pais, irm\~{a}s e fam\'{i}lia.

% Aos meus colegas da PUC--Rio, quem me fizeram adorar esse lugar.

% Aos professores Marcos da Silvera, Jean--Marie Nicolas e Anne Germa que me ofereceram a oportunidade desta coopera\c{c}\~{a}o.

% Ao pessoal do departamento de Matem\'{a}tica para a ajuda de todos os dias, em particular \`{a} Ana Cristina, Creuza e ao Sinesio.
% }%
%
%\renewcommand{\acknowledgment}{}


%%%%%%%%%%%%%%%%%%%%%%%%%%%%%%%%%%%%%%%%%%%%%%%%%%%%%%%%%%%%%%%%%%%%%%%%%%%%%%%%

\chaves{%
  \chave{Ambientes de experimentação em rede}%
  \chave{Programação orientada a eventos}%
  \chave{Passagem de mensagens por filas}
}

\resumo{
Neste trabalho será implementado um controlador de experimentos OMF na linguagem de programação Céu que possibilite o uso dos construtos nativos da linguagem para tratamento de eventos na descrição de experimentos destinados a ambientes de experimentação em rede. Um sub-produto deste trabalho será um \textit{binding} em Céu para uma biblioteca de AMQP escrita na linguagem de programação C, que concilie as propriedades e métodos desta biblioteca ao paradigma de programação reativa e estruturada de Céu.
}


%%%%%%%%%%%%%%%%%%%%%%%%%%%%%%%%%%%%%%%%%%%%%%%%%%%%%%%%%%%%%%%%%%%%%%%%%%%%%%%%

\chavesuk{
  \chave{Networking Testbeds}%
  \chave{Event-oriented Programming}%
  \chave{Message Queuing}
}

\resumouk{%
In this project an OMF experiment controller will be implemented in the Céu programming language to enable the use of the languages's native event handling constructs for the description of experiments targeting networking testbeds. A byproduct to this work will be a Céu binding to an AMQP library written in C that combines the properties and methods of such library with the structured reactive programming paradigm of Céu. 
}


%%%%%%%%%%%%%%%%%%%%%%%%%%%%%%%%%%%%%%%%%%%%%%%%%%%%%%%%%%%%%%%%%%%%%%%%%%%%%%%%

% \modotabelas{figtab} % nada, fig, tab ou figtab
\renewcommand{\modotabelas}{}

%%%%%%%%%%%%%%%%%%%%%%%%%%%%%%%%%%%%%%%%%%%%%%%%%%%%%%%%%%%%%%%%%%%%%%%%%%%%%%%%

\begin{document}


\chapter{Introdu\c{c}\~ao}

A emerg\^encia de ambientes de experimenta\c{c}\~ao, conhecidos como \textit{testbeds}, para o amparo do desenvolvimento de agentes conectatos em rede, cria a necessidade por protocolos e ferramentas que possibilitem a descri\c{c}\~ao dos recursos, da configura\c{c}\~ao e dos processos que definem completamente um experimento reproduz\'ivel. Neste contexto, desenvolveu-se o \textit{OMF} \cite{Rakotoarivelo2010:OCM:1713254.1713267}, um arcabou\c{c}o para ger\^encia de \textit{testbeds} e controle de experimentos.

O objetivo do OMF \'e assegurar o n\'ivel apropriado de abstra\c{c}\~ao, tanto do ponto de vista do operador de \textit{testbeds} quanto do pesquisador \cite{OMFExecutiveSummary}. O operador tem acesso a um conjunto de servi\c{c}os que facilitam a ger\^encia do \textit{testbed}. Por outro lado, o arcabou\c{c}o oferece suporte a \textit{scripts}, escritos pelo pesquisador em linguagem de alto n\'ivel, que descrevem o experimento a ser realizado, automatizando sua execu\c{c}\~ao.

O LabLua desenvolveu, nos \'ultimos anos, pesquisas sobre redes de sensores sem fio (RSSF) e participou do desenvolvimento do \textit{testbed} CeuNaTerra \cite{rossettoceunaterra}, implementado com o apoio da Rede Nacional de Ensino e Pesquisa. Também desenvolveu um conjunto de tecnologias que podem ser empregadas em aplica\c{c}\~oes destinadas a RSSF, como a linguagem de programa\c{c}\~ao estruturada s\'incrona e reativa C\'eu \cite{ceu.tr} e o sistema de programa\c{c}\~ao de RSSF Terra \cite{Branco:2015:TFS:2782756.2811267}. 

Neste contexto, surgiu a ideia de se expandir os casos de uso da linguagem C\'eu atrav\'es do desenvolvimento nesta de uma implementa\c{c}\~ao alternativa do arcabou\c{c}o OMF. Isto se deve a experimentos OMF serem descritos em torno de eventos externos e internos aos agentes sob teste e a exist\^encia de construtos nativos na linguagem para defini\c{c}\~ao e tratamento de eventos como entidades de primeira classe.

% \begin{figure}
%   \includegraphics*[width=\linewidth]{ctor4_none.eps}
%   \caption{Uma figura}
% \end{figure}


\chapter{Situa\c{c}\~ao Atual}

O arcabou\c{c}o OMF pode ser discretizado em tr\^es planos: controle, medi\c{c}\~ao e gerenciamento \cite{Rakotoarivelo2010:OCM:1713254.1713267}. O primeiro engloba as ferramentas e diretivas que possibilitam ao pesquisador definir e por em execu\c{c}\~ao um experimento; aqui destaca-se o \textit{controlador de experimentos (CE)}. O segundo \'e constitu\'ido das tecnologias que possibilitam ao pesquisador coletar m\'etricas dos recursos sob teste para avalia\c{c}\~ao posterior; a coleta de m\'etricas \'e definida atrav\'es da \textit{OMF Measurement Library (OML)}, n\~ao abordada neste trabalho. Finalmente, a camada de gerenciamento diz respeito a infraestrutura e \textit{testbed} nos quais o experimento \'e executado, destacando-se neste plano os recursos e seus controladores.

Os recursos na camada de gerenciamento s\~ao os alvos do experimento. Para possibilitar o desenvolvimento de tais recursos como m\'odulos coesos e independentes, o arcabou\c{c}o OMF define entidades denominadas \textit{controladores de recursos (CRs)}, que s\~ao respons\'aveis por intermediar a comunica\c{c}\~ao entre o CE e os recursos sob seu controle. A distribui\c{c}\~ao oficial do OMF disponibiliza uma biblioteca em \textit{Ruby}, chamada de \textit{omf\_rc} \cite{omf.rc}, que permite a descri\c{c}\~ao e uso de controladores de recursos. Contudo, controladores de recursos podem ser implementados sem depender desta biblioteca, desde que lidem corretamente com os protocolos de comunica\c{c}\~ao.

O CE na camada de controle \'e o aplicativo que executa o experimento, sendo sua implementa\c{c}\~ao oficial chamada de \textit{omf\_ec} \cite{omf.ec}. Para executar um experimento, o pesquisador deve fornecer como entrada para o CE um arquivo denominado \textit{descri\c{c}\~ao de experimento (DE)} que especifique completamente os recursos sob teste, sua configura\c{c}\~ao inicial e as a\c{c}\~oes a serem executadas sobre os mesmos durante o experimento. Com base neste arquivo, o CE envia pedidos aos CRs dos recursos sob teste e coleta as respostas enviadas por estes, exibindo-as ao usu\'ario caso este especifique tal comportamento. Na distribui\c{c}\~ao oficial do OMF, tais arquivos devem ser elaborados na linguagem de programa\c{c}\~ao \textit{OMF Experiment Description Language (OEDL)} \cite{OEDL}. Esta linguagem \'e, na verdade, uma extens\~ao de \textit{Ruby} que prov\^e uma s\'erie de comandos espec\'ificos ao dom\'inio de defini\c{c}\~ao e orquestra\c{c}\~ao de experimentos. Deste modo, o pesquisador pode facilmente descrever seus experimentos, mas tamb\'em tem acesso as funcionalidade nativas de \textit{Ruby}, sendo poss\'ivel realizar l\'ogicas mais complexa em seu DE.

A comunica\c{c}\~ao entre o CE e os CRs n\~ao se d\'a de forma direta. O arcabou\c{c}o OMF exige o emprego de um servidor que suporte o paradigma de troca de mensagens \textit{pub/sub}. Utilizando-se este servidor, ambas as partes trocam mensagens atrav\'es de t\'opicos definidos ao longo da execu\c{c}\~ao do experimento, segundo o protocolo \textit{Federated Resource Control Protocol (FRCP)} \cite{frcp}. Este protocolo especifica o formato do \textit{payload} das mensagens intercambiadas e uma s\'erie de regras sobre como o interc\^ambio destas deve ocorrer, segundo seu tipo e par\^ametros. A implementa\c{c}\~ao oficial de OMF suporta dois tipos de protocolos de troca de mensagens: \textit{Extensible Messaging and Presence Protocol (XMPP)} e \textit{Advanced Message Queuing Protocol (AMQP)} \cite{oasis2012advanced}, este mais recomendado.

AMQP especifica que a troca de mensagens entre aplica\c{c}\~oes deve ocorrer por interm\'edio de filas de mensagens atreladas a \textit{exchanges}, ambos definidos e gerenciados por um servidor central, chamado de \textit{broker}. Aplica\c{c}\~oes produtoras de mensagens as enviam para um \textit{exchange} que, com base em seu tipo e em uma chave de roteamento possivelmente definida no cabeçalho de cada mensagem, roteia cada uma para o subconjunto apropriado de filas ou para o descarte, caso nenhuma fila atrelada ao \textit{exchange} esperasse por mensagens com a chave especificada. Aplica\c{c}\~oes consumidoras, por sua vez, increvem-se em filas e consomem mensagens destas segundo um crit\'erio de \textit{round-robin}. No contexto de OMF deve-se utilizar um servidor que implemente o protocolo AMQP, através do qual o CE e CRs realizam sua troca de mensagens.

O ciclo de vida de um experimento inicia-se com a elabora\c{c}\~ao de uma descri\c{c}\~ao de experimento que \'e ent\~ao passada como entrada a um controlador de experimentos. Este, por sua vez, inicia uma sequ\^encia de troca de mensagens com os controladores de recursos especificados na DE. Esta troca de mensagens ocorre atrav\'es de um \textit{broker} AMQP, sendo a defini\c{c}\~ao dos t\'opicos, empacotamento das mensagens e roteamento destas realizados segundo as regras estipuladas pelo FRCP. Tais processos s\~ao executados por tr\'as dos panos pelo CE, sem a necessidade de conhecimento ou interven\c{c}\~ao do pesquisador sobre os mesmos. Ao fim do experimento, o estado no servidor de mensagens \'e limpo e poss\'iveis m\'etricas s\~ao armazenadas automaticamente.

Embora OMF seja um arcabou\c{c}o robusto, uma limita\c{c}\~ao inerente a \textit{Ruby} \'e a aus\^encia de eventos como entidades de primeira classe. Isto acarreta em defini\c{c}\~oes de eventos e diretivas para seu tratamento n\~ao t\~ao naturais, constituindo uma barreira para a introdu\c{c}\~ao de pesquisadores a este ambiente. Para facilitar o uso de \textit{testbeds} OMF, pretendemos oferecer a possibilidade do pesquisador escrever a descri\c{c}\~ao de seu experimento em C\'eu, uma linguagem desenvolvida para lidar com sistemas orientados a eventos.


\chapter{Proposta e Objetivos do Trabalho}

O principal produto deste trabalho ser\'a a implementa\c{c}\~ao de um CE alternativo em C\'eu que apresente paridade de funcionalidades frente ao oficial e seja capaz de processar DEs escritas em C\'eu. A escrita de DEs em C\'eu tornar\'a poss\'ivel a defini\c{c}\~ao e tratamento de eventos como entidades de primeira classe, simplificando a estrutura desses arquivos. Al\'em disso,introduzir\'a um novo caso de uso de C\'eu sob o dom\'inio de defini\c{c}\~ao e orquestra\c{c}\~ao de experimentos em \textit{testbeds}. 

\'E necess\'aria a ado\c{c}\~ao de uma biblioteca de AMQP para que o CE C\'eu seja capaz de se comunicar com os CRs segundo o processo delineado no cap\'itulo anterior. Sua implementa\c{c}\~ao em C\'eu como um m\'odulo independente justifica-se por tornar mais conveniente seu emprego em futuras aplica\c{c}\~oes C\'eu. Esta biblioteca ser\'a implementada com base na biblioteca C \textit{rabbitmq-c} \cite{rabbitmq.c}, expondo as funcionalidades essenciais do AMQP por esta suportadas. Contudo, devido a natureza bloqueante da implementa\c{c}\~ao de certas funcionalidades na biblioteca C, propriedade esta contr\'aria ao paradigma s\'incrono reativo de C\'eu, ser\~ao impostas limita\c{c}\~oes ao uso de tais funcionalidades para que se ofere\c{c}a uma biblioteca de mais alto n\'ivel em C\'eu.

As bibliotecas desenvolvidas adotarão de forma o mais rigorosa quanto possível as imposições de Céu, em detrimento aos conceitos dos protocolos AMQP e OMF. Por exemplo, ao invés de se definir uma variável de estado em uma chamada AMQP que destrói uma entidade quando do término de uma conexão, será imposto ao programador que a destruição da entidade ocorra através do término do escopo no qual sua representação em Céu reside. Deste modo, impera o conceito de programação estruturada de Céu: o fluxo de execução do programa se dá segundo reações a eventos, as quais implicam a ativação ou terminação de escopos que, por sua vez, levam a criação e execução ou terminação de entidades (e.g. filas AMQP).

Para valida\c{c}\~ao dos produtos desenvolvidos, ser\'a disponibilizada uma s\'erie de testes que garanta a corretude das funcionalidades da biblioteca AMQP e do CE OMF C\'eu. Finalmente, se houver disponibilidade de tempo, tentar-se-á também neste trabalho apresentar uma demonstração de um experimento que execute sobre o \textit{testbed} CeuNaTerra atrav\'es de comandos publicados pelo CE desenvolvido em C\'eu. Ainda, para fins de comparaç\~ao, seria tamb\'em feita uma demonstra\c{c}\~ao equivalente utilizando o CE oficial.

Este trabalho n\~ao ir\'a oferecer uma integra\c{c}\~ao com OML, mas o c\'odigo final do CE dever\'a ser o mais claro e extens\'ivel poss\'ivel para que tal integra\c{c}\~ao possa vir a ser adicionada em trabalhos futuros. Al\'em disso, n\~ao ser\'a entregue uma biblioteca em C\'eu para implementa\c{c}\~ao de controladores de recursos.


\chapter{Atividades Realizadas}

\section{Estudos Preliminares}

O aluno tinha prévia experiência com Céu, devido a participação em projeto de pesquisa para desenvolvimento de algoritmos distribuídos destinados a ambiente de redes de sensores sem fio. Contudo, tal pesquisa ocorreu ao longo de 2012 e a linguagem sofreu profundas mudanças desde então. Durante esta pesquisa o aluno também foi exposto ao conceito de \textit{testbeds}, mas não chegou a utilizá-las.

Por outro lado, neste trabalho o aluno foi exposto pela primeira vez aos conceitos, técnicas e ferramentas dos domínios de envio de mensagens por fila em sistemas distribuídos, empregados no AMQP, e orquestragem de ambientes de experimentação, empregados no OMF. 

\section{Estudos Conceituais e da Tecnologia}

Em razão da exposição limitada do aluno aos domínios dos protocolos AMQP e OMF, foi necessário um estudo aprofundado de suas documentações. Neste processo, estudou-se também o servidor \textit{RabbitMQ}, uma instância de um \textit{broker} AMQP, e as implementações oficiais em \textit{Ruby} de um controlador de experimentos OMF e de uma biblioteca para desenvolvimento de controladores de recursos OMF.

Através do estudo do material relacionado, o aluno familiarizou-se com as entidades primitivas de AMQP, aprendeu a forma como são expostas através do \textit{RabbitMQ} e as propriedades deste para ser capaz de ajustar suas configurações a fim de realizar os testes necessários. Por fim, analisou bibliotecas de AMQP em \textit{JavaScript}, \textit{Python} e \textit{C}, a fim de ter uma boa base para projetar e implementar a versão em Céu.

No contexto de OMF, estudou-se a arquitetura de um projeto OMF, a fim de se entender bem o encaixe de cada componente da mesma, focando-se principalmente nos papéis do controlador de experimentos e do servidor de mensagens utilizado para a comunicação entre os controladores de recursos e o controlador de experimentos. Ainda quanto ao processo de comunicação, o aluno dominou os conceitos e regras estipulados pelo FRCP, que dita como a comunicação deve ocorrer.

Por fim, foi necessário ao aluno ler o manual e estudar os tutoriais de Céu, a fim de se refamiliarizar com a linguagem. Neste sentido, também foi importante adaptar-se ao estilo de programação reativa e estruturada imposto por Céu, que difere significativamente de linguagens as quais estava mais acostumado, como \textit{C}, \textit{Java} e \textit{JavaScript}.

\section{Testes e Protótipos para Aprendizado e Demonstração}

O principal desenvolvimento se deu na frente de AMQP. Implementou-se uma versão inicial de uma biblioteca de AMQP em Céu \cite{ceu.rabbitmq}, que disponibiliza módulos equivalentes às entidades primitivas de AMQP além de auxiliares para certas operações, como o envio de mensagens. Para cada módulo desenvolveram-se testes simples que validam sua funcionalidade. Por fim, implementaram-se dois exemplos empregando-se todos os módulos, a fim de se realizar um teste de integração completo. Em um exemplo enviou-se uma mensagem para um consumidor de mensagens desenvolvido na biblioteca de \textit{Python}. No outro fez-se o caminho contrário: recebeu-se uma mensagem de um publicador escrito em \textit{Python}. Assim, a elaboração desta biblioteca seguindo boas práticas de Céu comprovou a viabilidade da linguagem para sua aplicação sob este domínio.

Sob o escopo de OMF, além dos estudos realizados, executou-se um experimento simples sobre um controlador de recursos simulado utilizando-se a implementação em \textit{Ruby} de um controlador de experimentos. Este teste possibilitou ao aluno aprender o funcionamento da ferramenta, observar o fluxo da execução de um experimento e compreender a troca de mensagens que ocorre entre os dois pontos ao longo deste processo. 

\section{Método}

O desenvolvimento de cada parte do projeto, isto é, os componentes AMQP e OMF, se deu em duas etapas: primeramente, um estudo do problema a ser atacado e, posteriormente, a implementação da solução. 

O estudo dos conceitos, técnicas e ferramentas dos domínios de AMQP e OMF foi feito de forma bastante aprofundada, tendo sido os pontos mais relevantes documentados e apresentados para a orientadora. Os resumos da documentação foram cuidadosamente registrados para servirem de material de referência durante o desenvolvimento dos códigos do projeto.

O desenvolvimento dos códigos foi um processo bastante iterativo. Conforme os módulos foram desenvolvidos, consultou-se o autor da linguagem Céu para sua análise e crítica, com base na qual os módulos passaram pela refatoração necessária. Deste modo, foi possível o desenvolvimento de módulos fiéis ao paradigma de programação reativa e estruturada imposto pela linguagem e nos quais empregam-se as funcionalidades mais recentes de Céu, servindo assim também de um teste para as mesmas.



\chapter{Plano de A\c{c}\~ao}

\section{Proposta Original}

Fez-se um estudo entre os meses de Janeiro e Março focado nas partes fundamentais do trabalho. Primeiramente, estudou-se AMQP e RabbitMQ (uma implementa\c{c}\~ao popular de \textit{broker} AMQP), focando-se tanto no projeto de arquitetura quanto em sessões pr\'aticas segundo tutorais. Posteriormente, estudou-se o arcabou\c{c}o OMF, de modo a tamb\'em se entender o projeto de arquitetura deste e como se encaixam, especificamente, o controlador de experimentos, os controladores de recursos e o FRCP. Com base nestes fundamentos dar-se-\'a partida no desenvolvimento do projeto ao longo do ano.

No que tange a biblioteca de AMQP em C\'eu, foi proposta a implementação dos m\'etodos definidos pelo protocolo e dos respectivos testes que validassem seu correto funcionamento. Al\'em disso, propôs-se um estudo de demais bibliotecas AMQP mirando-se paridade de funcionalidades. Quanto ao controlador de experimentos OMF, seria primeiramente estudada em detalhes sua implementa\c{c}\~ao oficial em \textit{Ruby}. Com base nisto, seria desenvolvida uma s\'erie de m\'odulos e testes associados, a fim de se compor o CE OMF C\'eu. Assim, a implementa\c{c}\~ao em C\'eu proposta seria o mais robusta e extens\'ivel poss\'ivel.

Na primeira quinzena de Maio, deveria ter sido entregue a implementa\c{c}\~ao de uma vers\~ao inicial da biblioteca de AMQP em C\'eu, na qual estariam implementados os principais m\'etodos AMQP com respetivos testes de unidade. Esta biblioteca seria continuamente aprimorada ao longo do ano, principalmente atrav\'es da corre\c{c}\~ao de \textit{bugs} e melhorias no c\'odigo, mas tamb\'em, possivelmente, atrav\'es da adi\c{c}\~ao de extens\~oes do RabbitMQ.

Ao longo de Maio pretendia-se desenvolver a base do CE OMF em C\'eu, especificamente os componentes para defini\c{c}\~ao de propriedades e aplica\c{c}\~oes de um experimento. Al\'em disso, tamb\'em deveria ter sido produzido um m\'odulo para o FRCP. Em Junho, teria-se estendido o CE com a adi\c{c}\~ao de grupos e eventos. Ao final de Junho, deveria ser poss\'ivel ao pesquisador executar um experimento utilizando todos os recursos aqui listados, mesmo que ainda estivessem presentes limita\c{c}\~oes.

\section{Realizações em Projeto Final I}

Ao longo do primeiro período constatou-se que a proposta original fora um tanto ambiciosa, dadas as limitações de tempo impostas ao aluno devido a participação em um elevado número de disciplinas. Contudo, produziu-se uma ampla documentação sobre os conceitos, regras e implementações das tecnologias AMQP e OMF e desenvolveu-se parte significativa da implementação da biblioteca Céu para AMQP.

A principal realização foi definitivamente o desenvolvimento da biblioteca AMQP em Céu. Isto exigiu uma familiarização as peculiaridades da linguagem e uma reflexão sobre como melhor conciliar os conceitos de AMQP com o modelo de programação estruturada que impera em Céu. Por ser ainda uma linguagem em desenvolvimento, a implementação de alguns módulos esbarrou em \textit{bugs} do compilador; estes, todavia, foram prontamente corrigidos pelo autor da linguagem.

Quanto ao OMF, embora fora planejado o desenvolvimento de um módulo limitado para controle de experimentos em Céu, isto não aconteceu. Contudo, estendeu-se o estudo do protocolo realizado no começo do ano através de uma análise da implementação em \textit{Ruby} do controlador de experimentos, o que possibilitou a compreensão da maneira como grupos e aplicações são representados, elucidando, por exemplo, a necessidade de gerenciar diferentes instanciações de uma mesma aplicação em um grupo; além disso, este estudo possibilitou uma melhor compreensão do encaixe do FRCP sob o escopo do controlador de experimentos. 

\section{Projeto Final II}

Em resumo, para o segundo período, os objetivo principais s\~ao: continuar o desenvolvimento do módulo Céu de AMQP, implementar o controlador de experimentos OMF em C\'eu, desenvolver uma demonstra\c{c}\~ao interessante para valida\c{c}\~ao do projeto, elimina\c{c}\~ao de \textit{bugs} e refatoramento do c\'odigo onde vantajoso e, por fim, elabora\c{c}\~ao de um detalhado relat\'orio final.

Para a biblioteca Céu de AMQP, faltam as seguintes pendências: automação dos testes de cada módulo; refatoração do módulo de filas para que as mensagens sejam recebidas por clientes da biblioteca através de instanciações de tal módulo; desenvolvimento dos exemplos oficiais utilizando o módulo Céu para facilitar seu aprendizado, focando em explicitar as diferenças de projeto relativas a bibliotecas AMQP escritas em outras linguagens, devido a imposição de conceitos e técnicas de programação estruturada em Céu.

Contudo, o controlador de experimentos OMF em Céu é a prioridade. Primeiramente, será desenvolvido um módulo base que permita a definição das entidades básicas: grupos e aplicações. Para isto, utilizar-se-á a integração de Céu a Lua, possibilitando a definição de tabelas para especificação das entidades. Estendendo-se este módulo base, serão implementadas as etapas iniciais definidas pelo FRCP, com base no estudo deste realizado ao longo de Projeto Final I, para que se verifique a troca inicial de mensagens ocorrida em um experimento entre um controlador de recursos e o controlador de experimentos OMF em Céu.

Feito isto, há de se estender o controlador de experimentos para aceitar um arquivo Céu mais elaborado que descreva as operações a serem executadas, como, por exemplo, a operação de inicialização das aplicações instanciadas dentro de um grupo. Nesta etapa, há de se implementar uma versão em Céu de exemplos oficiais a fim de se verificar a corretude do módulo desenvolvido.

Por fim, tentar-se-á atacar o problema de integração do controlador de experimentos ao \textit{testbed} CeuNaTerra, pelo menos a nível de estudo. Se possível, sua implementação será planejada e executada ao longo da disciplina de Projeto Final II.


\chapter{Cronogramas}

\section{Cronograma Original}

O cronograma abaixo resume o que havia sido especificado na Proposta:
\begin{itemize}  
\item Janeiro a Março: estudo de Céu, AMQP e OMF.
\item Abril: desenvolvimento da biblioteca de AMQP em Céu.
\item Maio: fim do desenvolvimento da biblioteca de AMQP e início do desenvolvimento da base do controlador de experimentos OMF em Céu.
\item Junho: fim do desenvolvimento de uma versão limitada do controlador de experimentos e documentação dos trabalhos.
\item Julho a Setembro: melhorias pontuais ao que fora desenvolvido.
\item Outubro: continuidade da implementação do controlador de experimentos, desenvolvimento de testes e início da integração ao \textit{testbed} CeuNaTerra.
\item Novembro: fim da implementação do controlador de experimentos, aprimoramentos aos códigos produzidos e fim de integração ao CeuNaTerra
\item Dezembro: escrita do relatório final.
\end{itemize}

\section{Cronograma Revisado}

O cronograma que segue especifica o que foi de fato desenvolvido e o trabalho que há de ser feito ao longo da disciplina de Projeto Final II:
\begin{itemize}  
\item Janeiro e Fevereiro: estudo de Céu, AMQP e OMF.
\item Março: estudo de FRCP e de biblioteca em \textit{Ruby} para criação de controladores de recursos OMF.
\item Abril e Maio: desenvolvimento da biblioteca de AMQP e estudo da implementação em \textit{Ruby} do controlador de experimentos OMF.
\item Junho: documentação dos trabalhos desenvolvidos ao longo do período.
\item Julho a Setembro:
  \begin{itemize}  
  \item Desenvolvimento do módulo base de OMF, permitindo a declaração de grupos e aplicações.
  \item Automação dos testes das biblioteca de AMQP em Céu.
  \item Refatoração do módulo de filas.
  \item Desenvolvimento dos exemplos oficiais do \textit{RabbitMQ}.
  \end{itemize}
\item Outubro: 
  \begin{itemize}  
  \item Continuidade da implementação do controlador de experimentos OMF em Céu.
  \item Desenvolvimento de testes e versões em Céu dos exemplos oficiais de OMF.
  \item Estudo da integração ao \textit{testbed} CeuNaTerra.
  \end{itemize}
\item Novembro: 
  \begin{itemize}  
  \item Fim do desenvolvimento do controlador de experimentos OMF em Céu.
  \item Aprimoramentos a todos os códigos produzidos (AMQP e OMF).
  \item Se possível, execução do projeto de integração ao \textit{testbed} CeuNaTerra do controlador de experimentos OMF em Céu.
  \end{itemize}
\item Dezembro: escrita do relatório final e apresentação para a banca.
\end{itemize}


\arial
\bibliography{FinalReport}
\addtocontents{toc}{\setcounter{tocdepth}{-1}}
\chapter*{Aceita\c{c}\~ao do Relatório de Projeto Final I}
\addtocontents{toc}{\setcounter{tocdepth}{2}}

\hfill\linebreak[4]
\hfill\linebreak[4]

\hfill
\hfill
\noindent\begin{tabular}{ll}
\makebox[3.5in]{\hrulefill} & \makebox[1.5in]{\hrulefill}\\
Assinatura do Aluno & Data\\[8ex]% adds space between the two sets of signatures
\makebox[3.5in]{\hrulefill} & \makebox[1.5in]{\hrulefill}\\
Assinatura da Orientadora & Data\\
\end{tabular}

\normalfont
%\begin{thenotations}
% %------------------------------------------------------------------------------%
% \section*{Simplicial Complex}
% %------------------------------------------------------------------------------%
% %
% \noindent
% \begin{tabular}{ll}
% \mathbb{R} & conjunto dos n\'{u}meros reais \\
% \end{tabular}
%
%\end{thenotations}
%\printindex

%\appendix

%\chapter{Primeiro Ap\^{e}ndice}
%O primeiro ap\^{e}ndice deve vir ap\'{o}s as refer\^{e}ncias bibliogr\'{a}ficas. Depois que voc\^{e} colocar a diretiva ``{$\backslash$}apendix'', todos os %``{$\backslash$}chapter\{\}'' v\~{a}o gerar ap\^{e}ndices.

\end{document}
